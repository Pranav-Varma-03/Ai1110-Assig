\documentclass{beamer}

%packages:
% \usepackage{tfrupee}
% \usepackage{amsmath}
% \usepackage{amssymb}
% \usepackage{gensymb}
% \usepackage{txfonts}

% \def\inputGnumericTable{}

% \usepackage[latin1]{inputenc}                                 
% \usepackage{color}                                            
% \usepackage{array}                                            
% \usepackage{longtable}                                        
% \usepackage{calc}                                             
% \usepackage{multirow}                                         
% \usepackage{hhline}                                           
% \usepackage{ifthen}
% \usepackage{caption} 
% \captionsetup[table]{skip=3pt}  
% \providecommand{\pr}[1]{\ensuremath{\Pr\left(#1\right)}}
% \providecommand{\cbrak}[1]{\ensuremath{\left\{#1\right\}}}
% %\renewcommand{\thefigure}{\arabic{table}}
% \renewcommand{\thetable}{\arabic{table}}      

\setbeamertemplate{caption}[numbered]{}

\usepackage{enumitem}
\usepackage{tfrupee}
\usepackage{amsmath}
\usepackage{amssymb}
\usepackage{graphicx}
\usepackage{txfonts}

\def\inputGnumericTable{}

\usepackage[latin1]{inputenc}                                 
\usepackage{color}                                            
\usepackage{array}                                            
\usepackage{longtable}                                        
\usepackage{calc}                                             
\usepackage{multirow}                                         
\usepackage{hhline}                                           
\usepackage{ifthen}
\usepackage{caption} 
\captionsetup[table]{skip=3pt}  
\providecommand{\pr}[1]{\ensuremath{\Pr\left(#1\right)}}
\providecommand{\cbrak}[1]{\ensuremath{\left\{#1\right\}}}
\renewcommand{\thefigure}{\arabic{table}}
\renewcommand{\thetable}{\arabic{table}}   
\newcommand*{\Comb}[2]{{}^{#1}C_{#2}}
\providecommand{\brak}[1]{\ensuremath{\left(#1\right)}}
\newcommand{\mydet}[1]{\ensuremath{\begin{vmatrix}#1\end{vmatrix}}}
\providecommand{\cbrak}[1]{\ensuremath{\left\{#1\right\}}}
\providecommand{\sbrak}[1]{\ensuremath{{}\left[#1\right]}}
% Theme choice:
\usetheme{CambridgeUS}

% Title page details: 
\title{Assignment 7} 
\author{Pericherla Pranav Varma\\CS21BTECH11044}
\date{\today}
% \logo{\large \LaTeX{}}


\begin{document}

    % Title page
    \begin{frame}
        \titlepage 
    \end{frame}

    % Outline
    \begin{frame}{Outline}
        \tableofcontents
    \end{frame}

    \section{Question}
    	\begin{frame}{Question}
    	Papoulis Pillai Ch5 Ex 6-69:\\[9pt]
    Show that, if random variables $x$ and $y$ are $N(0,0,{\sigma_1}^2,{\sigma_2}^2,r)$ then,\\[6pt]
   	\begin{align*}
   	E \cbrak{ \mydet {xy} } = \dfrac{2}{\pi} \int_{0}^{c} arcsin \dfrac{\mu}{\sigma_1 \sigma_2} d \mu + \dfrac{2 \sigma_1 \sigma_2}{\pi} = \dfrac{2 \sigma_1 \sigma_2}{\pi} (cos \alpha + \alpha sin \alpha)
   	\end{align*}
    % Add the table of probability here,
	where, $r = sin \alpha $ and $C = r \sigma_1 \sigma_2.$
  
    	\end{frame}
%    
	\section{Theory}
	\begin{frame}{Theory}
	Given two jointly normal random variables $x$ and $y$, we form the mean 		
		\begin{align*}
		I = E \cbrak{g(x,y)} = \int_{- \infty}^{\infty} \int_{- \infty}^{\infty} g(x,y) f(x,y) dx dy
		\end{align*}
		of some function g(x, y) of (x, y). The above integral is a function $I(\mu)$ of the covariance
$\mu$ of the random variables $x$ and $y$ and of four parameters specifying the joint density
$f(x, y)$ of $x$ and $y$. We shall show that if $g(x, y)f(x, y) \longrightarrow 0$ as $(x, y) \longrightarrow \infty$, then 
		\begin{align*}
		\dfrac{\partial^n I(\mu)}{\partial {\mu}^n} = \int_{- \infty}^{\infty} \int_{- \infty}^{\infty} \dfrac{\partial^{2n} g(x,y)}{\partial x^n \partial y^n} f(x,y) dx dy = E \brak{\dfrac{\partial^{2n} g(x,y)}{\partial x^n \partial y^n}}
		\end{align*}
	\end{frame}
	
    \section{Solution}
        \begin{frame}{Solution}
        Using Price's theorem,
        \begin{align}
        \dfrac{\partial E \cbrak{ \mydet{xy}}}{\partial \mu} &= E \cbrak{ \dfrac{d \mydet{x}}{dx} + \dfrac{d \mydet{y}}{dy}} = E \cbrak{sgnx sgny}\\[9pt]
        &= P \cbrak{xy > 0} - P \cbrak{xy < 0} \\[9pt]
        &= \dfrac{2 \alpha}{\pi}\\[9pt] 	
        &=\dfrac{2}{\pi} arcsin{\dfrac{\mu}{\sigma_1 \sigma_2}}
        \end{align}
        if $\mu = 0$, then the RVs $x$ and $y$ are independent,\\[9pt] hence,\\
       
        \end{frame}
    %
    \begin{frame}{Solution}
    at $\mu = 0$
	\begin{align*}
	 E \cbrak{ \mydet{xy}} &= E \cbrak{\mydet x} E \cbrak{\mydet x}\\[9pt]
	  &= \dfrac{2}{\pi} \sigma_1 \sigma_2
	\end{align*}	    
Integrating (4) using the above,we obtain
	\begin{align*}
	E \cbrak{ \mydet{xy}} &= \dfrac{2}{\pi} \int_{ 0}^{\mu} arcsin{\dfrac{\mu}{\sigma_1 \sigma_2}} + \dfrac{2}{\pi} \sigma_1 \sigma_2 \\[9pt]
	&= \dfrac{2}{\pi} \sigma_1 \sigma_2 (cos{\alpha} + \alpha sin{\alpha})
	\end{align*}
 
    \end{frame}
\end{document}