\documentclass[journal,12pt,twocolumn]{IEEEtran}
\parindent 0px
\usepackage{setspace}

\usepackage{enumitem}
\usepackage{amsmath}

\title{Assignment 1 \\ \Large AI1110: Probability and Random Variables \\ }
\author{Pericherla Pranav Varma \\ \normalsize CS21BTECH11044 \\ \vspace*{20pt} \normalsize  06 April 2022 \\ \vspace*{20pt} \Large ICSE 2017 Grade 10}
\begin{document}
\maketitle
\doublespacing
Question 3(c):\\
The marks of 10 students of a class in an examination arranged in ascending order
is as follows:\\
13, 35, 43, 46, x, x+4, 55, 61, 71, 80.\\
If the median marks is 48, find the value of x. Hence find the mode of the given
data.\\[9pt]
Given: "data in ascending order"
\begin{enumerate}[label=(\roman*)]
	\item 13, 35, 43, 46, $x$, $x+4$, 55, 61, 71, 80.
	\item median = 48
\end{enumerate}
Required: 
\begin{enumerate}[label=(\roman*)]	
	\item value of $x$.
	\item mode of the data.
\end{enumerate}
\textbf{Solution}:\\[8pt]
Median(definition): it is the middle number in a sorted ordered list of number. \\

i.e., if $n$ be the number of entries in given data\\
then median of the data is:\\
\begin{enumerate}[label=(\roman*)]
\item if $n$ = odd, Median = $(\frac{n}{2})^{th}$ element.\\
\item if ($n$ = even), Median = average of $(\frac{n}{2})^{th}$ and $(\frac{n+1}{2})^{th}$ elements.\\[6pt]
\end{enumerate}
As, Here the value of n is \underline{odd}.\\
	therefore,\\
	\begin{align}
	Median &= \dfrac{(x)+(x+4)}{2}.\\
	  48 &= \dfrac{(x)+(x+4)}{2}.\\
	  48 &= x+2.
	\end{align}	
	therefore, $ \underline{x=46.}$ \\[10pt]
Requirement (ii) : mode of the data. \\[4pt]
mode(definition) : it is the most repeating number in the list of numbers.\\[2pt]
Given 	Data = \\13, 35, 43, \underline{46}, \underline{46}, 50, 55, 61, 71, 80.\\(after updating $x$)\\
and hence, the \underline{mode = 46.}\\
\underline{Verification}:
\begin{enumerate}[label=(\roman*)]
	\item obtained value of x using median is verified in python code
	\item calculated value of mode is verified using C code
\end{enumerate}
\end{document}
