\let\negmedspace\undefined
\let\negthickspace\undefined
%\RequirePackage{amsmath}
\documentclass[journal,12pt,twocolumn]{IEEEtran}
%
\usepackage{setspace}
 \usepackage{gensymb}
%\doublespacing
 \usepackage{polynom}
%\singlespacing
%\usepackage{silence}
%Disable all warnings issued by latex starting with "You have..."
%\usepackage{graphicx}
\usepackage{amssymb}
%\usepackage{relsize}
\usepackage[cmex10]{amsmath}
%\usepackage{amsthm}
%\interdisplaylinepenalty=2500
%\savesymbol{iint}
%\usepackage{txfonts}
%\restoresymbol{TXF}{iint}
%\usepackage{wasysym}
\usepackage{amsthm}
%\usepackage{pifont}
%\usepackage{iithtlc}
% \usepackage{mathrsfs}
% \usepackage{txfonts}
 \usepackage{stfloats}
% \usepackage{steinmetz}
 \usepackage{bm}
% \usepackage{cite}
% \usepackage{cases}
% \usepackage{subfig}
%\usepackage{xtab}
\usepackage{longtable}
%\usepackage{multirow}
%\usepackage{algorithm}
%\usepackage{algpseudocode}
\usepackage{enumitem}
 \usepackage{mathtools}
 \usepackage{tikz}
% \usepackage{circuitikz}
% \usepackage{verbatim}
%\usepackage{tfrupee}
\usepackage[breaklinks=true]{hyperref}
%\usepackage{stmaryrd}
%\usepackage{tkz-euclide} % loads  TikZ and tkz-base
%\usetkzobj{all}
\usepackage{listings}
    \usepackage{color}                                            %%
    \usepackage{array}                                            %%
    \usepackage{longtable}                                        %%
    \usepackage{calc}                                             %%
    \usepackage{multirow}                                         %%
    \usepackage{hhline}                                           %%
    \usepackage{ifthen}                                           %%
  %optionally (for landscape tables embedded in another document): %%
    \usepackage{lscape}     
% \usepackage{multicol}
% \usepackage{chngcntr}
%\usepackage{enumerate}

%\usepackage{wasysym}
%\newcounter{MYtempeqncnt}
\DeclareMathOperator*{\Res}{Res}
\DeclareMathOperator*{\equals}{=}
%\renewcommand{\baselinestretch}{2}
\renewcommand\thesection{\arabic{section}}
\renewcommand\thesubsection{\thesection.\arabic{subsection}}
\renewcommand\thesubsubsection{\thesubsection.\arabic{subsubsection}}

\renewcommand\thesectiondis{\arabic{section}}
\renewcommand\thesubsectiondis{\thesectiondis.\arabic{subsection}}
\renewcommand\thesubsubsectiondis{\thesubsectiondis.\arabic{subsubsection}}

% correct bad hyphenation here
\hyphenation{op-tical net-works semi-conduc-tor}
\def\inputGnumericTable{}                                 %%
\parindent 0px
\lstset{
%language=C,
frame=single, 
breaklines=true,
columns=fullflexible
}
%\lstset{
%language=tex,
%frame=single, 
%breaklines=true
%}
\usepackage{tfrupee}
\usepackage{enumitem}
\usepackage{amsmath}
\usepackage{amssymb}


\title{Assignment 1 \\ \Large AI1110: Probability and Random Variables \\ }
\author{Pericherla Pranav Varma \\ \normalsize CS21BTECH11044 \\ \vspace*{20pt} \normalsize  06 April 2022 \\ \vspace*{20pt} \Large ICSE 2017 Grade 10}
\begin{document}
\maketitle
\doublespacing
Question 3(c):\\
The marks of 10 students of a class in an examination arranged in ascending order
is as follows:\\
13, 35, 43, 46, x, x+4, 55, 61, 71, 80.\\
If the median marks is 48, find the value of x. Hence find the mode of the given
data.\\[9pt]
Given: "data in ascending order"
\begin{enumerate}[label=(\roman*)]
	\item 13, 35, 43, 46, $x$, $x+4$, 55, 61, 71, 80.
	\item median = 48
\end{enumerate}
Required: 
\begin{enumerate}[label=(\roman*)]	
	\item value of $x$.
	\item mode of the data.
\end{enumerate}
Solution:\\[8pt]
Median(definition): it is the middle number in a sorted ordered list of number. \\
i.e., if $n$ be the number of entries in given data\\
then median of the data is:\\
if $n$ = odd, Median = $(\frac{n}{2})^{th}$ element.\\
else ($n$ = even), Median = average of $(\frac{n}{2})^{th}$ and $(\frac{n+1}{2})^{th}$ elements.\\[6pt]
As, Here the value of n is \underline{odd}.\\
	therefore,\\
	\begin{align}
	Median &= \dfrac{(x)+(x+4)}{2}.\\
	  48 &= \dfrac{(x)+(x+4)}{2}.\\
	  48 &= x+2.
	\end{align}	
	therefore, $ \underline{x=46.}$ \\[10pt]
Requirement (ii) : mode of the data. \\[4pt]
mode(definition) : it is the most repeating number in the list of numbers.\\[2pt]
Given 	Data = \\13, 35, 43, \underline{46}, \underline{46}, 50, 55, 61, 71, 80.\\(after updating $x$)\\
and hence, the \underline{mode = 46.}\\
\underline{Verification}:
\begin{enumerate}[label=(\roman*)]
	\item obtained value of x using median is verified in python code
	\item calculated value of mode is verified using C code
\end{enumerate}
\end{document}
