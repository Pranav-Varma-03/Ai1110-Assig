\documentclass{beamer}

%packages:
% \usepackage{tfrupee}
% \usepackage{amsmath}
% \usepackage{amssymb}
% \usepackage{gensymb}
% \usepackage{txfonts}

% \def\inputGnumericTable{}

% \usepackage[latin1]{inputenc}                                 
% \usepackage{color}                                            
% \usepackage{array}                                            
% \usepackage{longtable}                                        
% \usepackage{calc}                                             
% \usepackage{multirow}                                         
% \usepackage{hhline}                                           
% \usepackage{ifthen}
% \usepackage{caption} 
% \captionsetup[table]{skip=3pt}  
% \providecommand{\pr}[1]{\ensuremath{\Pr\left(#1\right)}}
% \providecommand{\cbrak}[1]{\ensuremath{\left\{#1\right\}}}
% %\renewcommand{\thefigure}{\arabic{table}}
% \renewcommand{\thetable}{\arabic{table}}      

\setbeamertemplate{caption}[numbered]{}

\usepackage{enumitem}
\usepackage{tfrupee}
\usepackage{amsmath}
\usepackage{amssymb}
\usepackage{graphicx}
\usepackage{txfonts}

\def\inputGnumericTable{}

\usepackage[latin1]{inputenc}                                 
\usepackage{color}                                            
\usepackage{array}                                            
\usepackage{longtable}                                        
\usepackage{calc}                                             
\usepackage{multirow}                                         
\usepackage{hhline}                                           
\usepackage{ifthen}
\usepackage{caption} 
\captionsetup[table]{skip=3pt}  
\providecommand{\pr}[1]{\ensuremath{\Pr\left(#1\right)}}
\providecommand{\cbrak}[1]{\ensuremath{\left\{#1\right\}}}
\renewcommand{\thefigure}{\arabic{table}}
\renewcommand{\thetable}{\arabic{table}}   
\providecommand{\brak}[1]{\ensuremath{\left(#1\right)}}

% Theme choice:
\usetheme{CambridgeUS}

% Title page details: 
\title{Assignment 4} 
\author{Pericherla Pranav Varma}
\date{\today}
% \logo{\large \LaTeX{}}


\begin{document}

    % Title page
    \begin{frame}
        \titlepage 
    \end{frame}

    % Outline
    \begin{frame}{Outline}
        \tableofcontents
    \end{frame}

    \section{Question}
    	\begin{frame}{Exercise 13.3.5}
    The random variable X has a probability distribution P(X) of the following form, where k is some number:\\[6pt]
    
    % Add the table of probability here,
   \begin{table}[H]
			\def\arraystretch{1.5}    		
    		\centering
				 \begin{tabular}{|c|c|c|c|}
	\hline
	$\pr{X=0}$ & $\pr{X=0}$ & $\pr{X=0}$ & $\pr{X=0}$ \\
	\hline
	k & 2k & 3k & 0 \\
	\hline
\end{tabular}\\[6pt]
    		\caption{Probability Distribution}
    			\label{Table 1}
	\end{table}
    \begin{enumerate}[label=(\alph{enumi})]
    		\item Determine the value of k.
    		\item Find $\pr{X<2},\pr{X \le 2}$ and $\pr{X \ge 2}$.
    \end{enumerate} 
    	\end{frame}
    \section{Solution(a)}
        \begin{frame}{Solution(a)}
        Given a Random Variable X with its probability distribution.\\[9pt]
        (a) As we know the sum of all the probabilities in a probability distribution of a random variable must be one. i.e.,\\
	\begin{align}
		\sum^{n}_{k=1} \pr{X=k} &= 1 
        \label{Eq1}
        \end{align}
    \end{frame}
    \begin{frame}{Solution (a)}
    Hence, by using \eqref{Eq1} the sum of probabilities of given table:
    \begin{align*}
    \Longrightarrow 1 &= k+2k+3k+0 \\
    \Longrightarrow 1 &= 6k\\
    k &= \dfrac{1}{6}
    \end{align*}
    \end{frame}
    \section{Solution (b)}
    \begin{frame}{Solution (b)}
    (b) (i) Now we have to find $\pr{X<2}$\\
    \begin{align*}
    \pr{X<2} &= \pr{X=0} + \pr{X=1}\\
    &= k + 2k\\
    &= 3k\\
    \therefore \pr{X<2} &= 3\times\dfrac{1}{6} = \dfrac{1}{2}
    \end{align*}
    \end{frame}
    %
    \begin{frame}{Solution (b)}
     (ii) Now we have to find $\pr{X \le 2}$\\
    \begin{align*}
    \pr{X \le 2} &= \pr{X=0} + \pr{X=1} + \pr{X<2}\\
    &= k + 2k + 3k\\
    &= 6k\\
    \therefore \pr{X \le 2} &= 6\times\dfrac{1}{6} = 1
    \end{align*}
    \end{frame}
%
\begin{frame}{Solution (b)}
     (iii) Now we have to find $\pr{X \ge 2}$\\
    \begin{align*}
    \pr{X \ge 2} &= \pr{X=2} + \pr{X>2}\\
    &= 3k + 0\\
    &= 3k.\\
   \therefore \pr{X \ge 2} &= 3\times\dfrac{1}{6} = \dfrac{1}{2}.
    \end{align*}
    \end{frame}
\end{document}